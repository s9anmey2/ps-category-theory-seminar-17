\def\pathToRoot{../../}\input{\pathToRoot headers/header_lecturenotes}

\usepackage{todonotes}
\usepackage{semantic}
\usepackage{mathpartir}

\bibliographystyle{plainurl}

\DeclareMathOperator{\core}{core}

% Theorem Environments
\newtheorem{definition}{Definition}
\newtheorem{example}{Example}
\newtheorem{lemma}{Lemma}

% Title page
\title{Presheaf Model for Dependent Types}

\author{Nikita Ziuzin}

\begin{document}

\maketitle

\section*{Dependent types}

Suppose you want to write a function that returns the head of a given list of
natural numbers, using the classical definition of the list type:
\[
  List = Nil : Unit \to List~|~Cons : \NN \times List \to List
\]
Writing the case for $Cons$ is easy --- just take the first element of the
pair.  But what about the case of an empty list? This case is tricky because it
is not clear what should the result be. Obviously, such function (if we do not
allow definition of partial functions which is the case in many programming
languages) cannot be well defined on the empty list without taking arbitrary
decisions. But maybe such problems can be prevent from happening at all?

A solution to this issue is an instance for the application of dependent types.
Dependent types are such types that include simple types, like products and
functions, that we have seen in simply-typed lambda calculus (STLC), but also
add a possibility to define types using values. With dependent types we would
be able to define a type for lists that includes the size of the list:
\[
  List(\NN) = Nil : Unit \to List(0)~|~Cons : \NN \times List(n) \to List(Suc(n))
\]
\begin{example}
  The function to construct a list of length $n$, having only $0$ as elements
  would then have the type $\Pi n: \NN.List(n)$, indicating that it accepts a
  natural number and returns a value of a type constructed using that number.
\end{example}
\begin{example}
  Our desired head function would have the type
  \[
    Hd: \Pi n: \NN.List(Suc(n)) \to \NN
  \]
  indicating that it cannot be applied to the empty list at all.
\end{example}

The important part is that the dependent typing reveals more information about
the function.

\section*{Formalization}

To formalize dependent types we will use a notion of typing context (which is
almost the same as typing environment as seen for STLC, with the exception that
it holds dependent types).

\begin{definition}[Context]
  A \emph{context} is a list of bindings that connect variables to types: $x_1: A_1,
  \dots, x_n: A_n$. A context is called well-formed if every variable in it is
  distinct from the other variables and each $A_i$ is a type in the context $x_1
  : A_1, \dots, x_{n-1} : A_{n-1}$, since our types may now include the variables
  from the context.
\end{definition}

Note that we extend the notion of a well-formed context from STLC --- in STLC
we only needed the variables to be distinct, since they had no influence on the
types.

Having contexts, we want to work with the following judgements: $a : A$ (term
$a$ has type $A$) and $A~type$ ($A$ is a type) and consider definitional
equality to be built-in into our formal system. That is, we want the types
$List(0)$ and $List(0 + 0)$ to be equal. So we also include the judgements
about the equality of terms $a = b : A$, types $A = B~type$, and contexts
$\Gamma = \Delta~ctxt$.

To put it together, we have six types of judgements:

\begin{align*}
  &\vdash \Gamma~ctxt && \text{$\Gamma$ is a valid context} \\
  &\Gamma \vdash A~type && \text{$A$ is a type in $\Gamma$} \\
  &\Gamma \vdash a : A && \text{$a$ is a term of type $A$ in context $\Gamma$} \\
  &\vdash \Gamma = \Delta && \text{$\Gamma$ and $\Delta$ are definitionally equal contexts} \\
  &\Gamma \vdash A = B && \text{$A$ and $B$ are definitionally equal types in context $\Gamma$} \\
  &\Gamma \vdash a = b : A && \text{$a$ and $b$ are definitionally equal terms of type $A$ in context $\Gamma$}
\end{align*}

We also need the structural rules:

\begin{definition}[Structural rules]
\begin{mathpar}
  \inferrule*[right=C-Emp]{ }{\vdash \diamond~ctxt}
  \and
  \inferrule*[right=C-Ext]{\Gamma \vdash A~type}{\vdash \Gamma, x : A~ctxt}
  \and
  \inferrule*[right=C-Ext-Eq]{\vdash \Gamma = \Delta~ctxt \\ \Gamma \vdash A = B~type}{\vdash \Gamma, x:A = \Delta, y:B~ctxt}
  \and
  \inferrule*[right=Var]{\vdash \Gamma, x:A,\Delta~ctxt}{\Gamma,x:A,\Delta \vdash x:A}
  \and
  \inferrule*[right=C-Eq-R]{\vdash \Gamma~ctxt}{\vdash \Gamma = \Gamma~ctxt}
  \and
  \inferrule*[right=C-Eq-S]{\vdash \Gamma = \Delta~ctxt}{\vdash \Delta = \Gamma~ctxt}
  \and
  \inferrule*[right=C-Eq-T]{\vdash \Gamma = \Delta~ctxt \\ \vdash \Delta = \Theta~ctxt}{\vdash \Gamma = \Theta~ctxt}
  \and
  \inferrule*[right=Tm-Conv]{\Gamma \vdash a : A \\ \vdash \Gamma = \Delta~ctxt \\ \Gamma \vdash A = B~type}{\Delta \vdash a : B}
  \and
  \inferrule*[right=Ty-Conv]{\vdash \Gamma = \Delta~ctxt \\ \Gamma \vdash A~type}{\Delta \vdash A~type}
  \and
  \inferrule*[right=Weak]{\Gamma, \Delta \vdash \mathcal{J} \\ \Gamma \vdash C~type}{\Gamma, x : C, \Delta \vdash \mathcal{J}}
  \and
  \inferrule*[right=Subst]{\Gamma, x: C, \Delta \vdash \mathcal{J} \qquad \Gamma \vdash c: C}{\Gamma, \Delta[c/x] \vdash \mathcal{J}[c/x]}
\end{mathpar}

where $x$ and $y$ in C-Ext and C-Ext-Eq are assumed to be fresh variables and
$\mathcal{J}$ in Weak and Subst ranges over judgements of the form $a : A,
A~type, a = b: A, A = B~type$. $[c/x]$ denotes the same substitution as seen in
STLC, except that it now also applies to the variables that may be contained in
types.
\end{definition}

The rules C-Emp and C-Ext give us the empty context ($\diamond$) and allow us
to extend the context (analogous to the environments from STLC).

C-Ext-Eq gives the meaning to the context equality: if we have two equal
contexts and one of them contains equal types, then the variables of these
types are the same in both contexts.

Var says that if a context contains a variable of a certain type, then the
judgement that this variable has that type in the context is satisfied.

Rules C-Eq define that context equality is an equivalence relation, very
similar rules for terms and types, stating that their equalities are also
equivalence relations, are omitted for brevity.

The rule Tm-Conv states that equivalent types and equivalent contexts may be
used in place of each other in judgements on terms, while the rule Ty-Conv
states that equivalent contexts can be used in place of each other in
judgements on types.

The rule Weak says that we can extend our context with a valid type, preserving
the validity of the typing judgements.

The rule Subst states that the validity of the typing judgements is preserved
under substitution.

There are two important dependent type formers to consider: dependent product
and dependent sum. The types formed with them have well-understood analogues in
set theory and are essential when working with dependent types. These are not
the only type formers that we can construct and we hope that the general
principles of their construction will be clear from these two examples.

To present these type formers we will be using formation, introduction,
elimination, and equality rules.

\subsection*{Dependent product}

Dependent product or dependent function type (often abbreviated as $\Pi$-type)
is a type of a function that returns a result which type may depend on the
function argument. It is defined by the following rules:

\begin{definition}[Dependent product]
  \begin{mathpar}
    \inferrule*[right=$\Pi$-F]{\Gamma \vdash A~type \\ \Gamma, x:A \vdash B~type}{\Gamma \vdash \Pi x:A. B~type}
    \and
    \inferrule*[right=$\Pi$-Eq]{\Gamma \vdash A_1 = A_2 ~type \\ \Gamma, x:A_1 \vdash B_1 = B_2 ~type}{\Gamma \vdash \Pi x:A_1. B_1 = \Pi x:A_2. B_2~type}
    \and
    \inferrule*[right=$\Pi$-I]{\Gamma, x:A \vdash b : B}{\Gamma \vdash \lambda x : A. b : \Pi x:A. B}
    \and
    \inferrule*[right=$\Pi$-I-Eq]
      {\Gamma, x:A_1 \vdash b_1 = b_2 : B_1 \\ \Gamma \vdash A_1 = A_2~type \\ \Gamma, x: A_1 \vdash B_1 = B_2~type}
      {\Gamma \vdash \lambda x : A_1. b^{B_1} = \lambda x : A_2. b^{B_2} : \Pi x:A_1. B_1}
    \and
    \inferrule*[right=$\Pi$-E]{\Gamma \vdash u : \Pi x:A. B \\ \Gamma \vdash a : A}{\Gamma \vdash App_{[x:A]B}(u, a): B[a / x]}
    \and
    \inferrule*[right=$\Pi$-C]{\Gamma \vdash \lambda x : A. b : \Pi x:A. B \\ \Gamma \vdash a : A}
    {\Gamma \vdash App_{[x:A]B}(\lambda x : A. b, a) = b[a / x]: B [a / x]}
  \end{mathpar}
\end{definition}

The rule $\Pi$-F serves for the formation of the type: if we know that having a
variable $x: A$ we obtain a type $B$, then we have our $\Pi$-type, making $x$
bound in that type.

The rule $\Pi$-Eq expresses that the definitional equality is respected by the
$\Pi$-former.

To introduce instances of $\Pi$-type we use the rule $\Pi$-I which looks almost
the same as in the simply-typed case, with the exception that our $\lambda$
here forms a dependent function type, not the simply-typed one.

The rule $\Pi$-I-Eq serves similar purpose to $\Pi$-Eq and ensures the
preservation of the definitional equality. The similar equality preservation
rules henceforth will be omitted for the other rules we introduce, but the
reader should keep their implicit presence in mind while seeing other term and
type formers that we introduce.

The rule $\Pi$-E regulates the application of a dependent function to some value
of the correct type results in the return value of a proper type. Note the rule
only concerns the resulting type here, not the value and that the bound
variable $x$ (shown as $[x : A] B$ in subscript) is replaced in the type, not
the value, which is handled by the next rule.

The rule $\Pi$-C defines the resulting value of a dependent function
application. It is almost the same $\beta$-conversion rule as in STLC, with the
variable substitution also occurring in the type.

The analogue to dependent product in set theory is a cartesian product over a
family of sets $\Pi_{i \in I}B_i$ where elements are the functions mapping an
index $i$ to an element of a corresponding set $B_i$.

\subsection*{Dependent sum}

The other very important type former is dependent sum type (also known as
dependent pair or $\Sigma$-type).

\begin{definition}[Dependent sum]
  \begin{mathpar}
    \inferrule*[right=$\Sigma$-F]{\Gamma \vdash A \text{ type}\\ \Gamma, x:A \vdash B \text{ type}}{\Gamma \vdash \Sigma x : A. B \text{ type}}
    \and
    \inferrule*[right=$\Sigma$-I]{\Gamma \vdash a : A\\ \Gamma\vdash b: B[a / x]}{\Gamma \vdash Pair_{[x:A]B} (a,b) : \Sigma x : A. B}
    \and
    \inferrule*[right=$\Sigma$-E]
      {\Gamma, z : \Sigma x: A. B \vdash C \text{ type} \\ \Gamma, x: A, y: B \vdash c : C[Pair_{[x:A]B}(x, y)/z] \\ \Gamma \vdash u : \Sigma x: A. B}
      {R^\Sigma_{[z : \Sigma x: A. B]C}([x : A, y: B]c, u): C[u/z]}
    \and
    \inferrule*[right=$\Sigma$-C]
      {\Gamma\vdash R^\Sigma_{[z : \Sigma x: A. B]C}([x : A, y: B]c, Pair_{[x:A]B}(a,b)): C[Pair_{[x:A]B}/z]}
      {\Gamma\vdash R^\Sigma_{[z : \Sigma x: A. B]C}([x : A, y: B]c, Pair_{[x:A]B}(a,b)) = c[a/x,b/y] : C [Pair_{[x:A]B}/z]}
  \end{mathpar}
\end{definition}

The formation rule $\Sigma$-F follows the same pattern as the rule for
dependent function, but the introduction is a bit different. Note that we take
such $b$, where the type contains a variable $x : A$ that is substituted by the
value $a$. Together the values $a$ and $b$ form a pair, where the type of the
second element depends on the value of the first. We use the notation $[x : A]
B$ here again to indicate that the variable $x$ is bound in the type $B$.

The rule $\Sigma$-E is again analogous to the one seen in dependent product: we
can replace a variable of a dependent pair type with a concrete term if he
have a term that contains that variable. And again we just say what happens to
the type, not the value there.

To define the effect on the values we use the rule $\Sigma$-C, which says that
the elements of the pair will replace the corresponding variables in the body
of the given term, but not the pair variable itself, such substitution occurs
only in the type.

We know the projection functions for the simply-typed products in STLC. Let us
now define the projection functions for the \emph{dependent} products, which we
call $\mathbf{p}$ for the first projection and $\mathbf{q}$ for the second
projection:
\[
  \mathbf{p}(a) := R^\Sigma_{[z : \Sigma x: A. B]A}([x : A, y: B]x, a) : A
\]
\[
  \mathbf{q}(a) := R^\Sigma_{[z : \Sigma x: A. B]B[z.1/x]}([x : A, y: B]y, a) : B[\mathbf{p}(a)/x]
\]
where $\Gamma \vdash A~type$, $\Gamma, x:A \vdash B~type$, and $\Gamma \vdash a
: \Sigma x: A. B$.

The analogue to dependent sum in set theory is a disjoint union. For every
family of sets $(B_i)_{i \in I}$ we can form the set $\Sigma_{i \in I} B_i :=
\{ (i, b)~|~i \in I \land b \in B_i\}$ whose elements consist of an index $i$
and an element of the corresponding set $B_i$.

It is worth notice that we may define the simply-typed products and functions
in terms of dependent types: we just need the types to not depend on each
other. So, assuming that $B$ does not depend on $A$, we get:
\[
  A \to B := \Pi x: A. B
\]
and
\[
  A \times B := \Sigma x: A. B
\]

\subsection*{Context morphism}

\begin{definition}[Context morphism]
  Let $\Gamma$ and $\Delta := x_1: A_1, \dots, x_n : A_n$ be valid contexts. If
  $\sigma := (a_1, \dots, a_n)$ is a sequence of $n$ valid terms we say that
  $\sigma \from \Gamma \to \Delta$ is a \emph{context morphism} if the
  following $n$ judgements hold:

  \begin{align*}
    &\Gamma \vdash a_1 : A_1\\
    &\Gamma \vdash a_2 : A_2[a_1/x_1]\\
    &\dots\\
    &\Gamma \vdash a_n : A_n[a_1/x_1][a_2/x_2]\dots[a_{n-1}/x_{n-1}]
  \end{align*}
\end{definition}

\begin{example}
  We always have the empty context morphism $()$ from any context $\Gamma$ to
  $\diamond$, it does not contain any terms and is the only context morphism
  from $\Gamma$ to $\diamond$.
\end{example}
\begin{example}
  We also have the identity context morphism $1$ for any $\Gamma = x_1 : A_1,
  \dots, x_n : A_n$, which is $(x_1, \dots, x_n)$.
\end{example}

A note on notation: we sometimes write $A\sigma$, where $\Gamma \vdash
A\sigma~type$, $\Delta \vdash A~type$, and $\sigma \from \Gamma \to \Delta$ for
some contexts $\Gamma$ and $\Delta$, to denote the application of $\sigma$ on
the right. That is, $A\sigma$ becomes here a type in the domain context of
$\sigma$, where $\sigma(A\sigma) = A$. The same notation is sometimes
analogously used for terms: $a\sigma$ to denote a term $\Gamma \vdash a\sigma :
A\sigma$, where $\Delta \vdash a : A$.

Having this notation, we also want the following equations to hold:
\[
  A1=A \quad (A\sigma)\tau = A(\sigma\tau) \quad a1=a \quad (a\sigma)\tau=a(\sigma\tau)
\]
For readability, we usually omit writing indices to the identity morphism $1$,
using it as the most suitable identity morphism for the surroundings. We also
assume correct types in the equations, i.e. $(A\sigma)\tau = A(\sigma\tau)$
assumes $\sigma \from \Delta \to \Gamma$, $\tau \from \Theta \to \Delta$, and
$\Gamma \vdash A~type$.

\begin{definition}[Context extension]
  If $\vdash \Gamma~ctxt$ and $\Gamma \vdash A~type$, there is a context
  $\Gamma, x:A$ (for brevity also written as $\Gamma.A$ from now on), a context
  morphism $\mathbf{p}: \Gamma.A \to \Gamma$, and a term $\Gamma.A \vdash
  \mathbf{q} : A \mathbf{p}$. This should satisfy the following property: for
  each $\vdash \Delta~ctxt$, a context morphism $\sigma \from \Delta \to
  \Gamma$, and $\Delta \vdash a : A \sigma$, there is a context morphism
  $(\sigma, a): \Delta \to \Gamma.A$ called \emph{context extension} such that:
  \[
    \mathbf{p}(\sigma, a) = \sigma \quad \mathbf{q}(\sigma, a) = a \quad (\sigma, a) \tau =
    (\sigma \tau, a \tau) \quad (\mathbf{p}, \mathbf{q}) = 1
  \]
  Note that in the last equation $1: \Gamma.A \to \Gamma.A$ and that we leave
  the indices on $\mathbf{p}$ implicit.
\end{definition}

\section*{Categories with Families}

There are several ways to model dependent types with category theory, the one
we show is using Categories with Families.

\begin{definition}[Category of families of sets]
  The category of families of sets, $\fcat{Fam}$, has as objects $(A, B)$ where
  $A$ is a set and $B = (B_a~|~a \in A)$ is an $A$-indexed family of sets $B_a$,
  $a \in A$. A morphism $(A, B) \to (A', B')$ is given by a pair $(f, g)$ where
  $f: A \to A'$ and $g$ is an $A$-indexed famiy of functions such that $g_a: B_a
  \to B_{fa}'$.
\end{definition}

\begin{definition}[Category with Families (CwF)]
A \emph{category with families} is given by ($\cat{C}$,$F$) where:

\begin{itemize}
  \item[$\bullet$] $\cat{C}$ is a category, whose

    objects: $\Gamma, \Delta, \dots$ are \emph{contexts}

    morphisms:  $\sigma, \tau, \dots$ are \emph{context morphisms}

  \item[$\bullet$] A terminal object $1$ in $\cat{C}$ is the empty context ($\diamond$).

  \item[$\bullet$] $F$ is a functor $F\from \cat{C}^\op \to \fcat{Fam}$; For a
    context $\Gamma$ we write $Ty_F(\Gamma)$ for the indexing set of the family
    $F(\Gamma)$ and its family as $(Ter_F(\Gamma;A)~|~A \in Ty_F(\Gamma))$;
    $\Gamma \vdash A~type$ means that $A \in Ty_F(\Gamma)$ and $\Gamma \vdash a
    : A$ means $a \in Ter_F(\Gamma;A)$. Note that for a context morphism
    $\sigma \from \Delta \to \Gamma$ the morphism $F(\sigma)$ acts on types
    $\Gamma \vdash A~type$ as $A\sigma$ and on terms $\Gamma \vdash a : A$ as
    $a\sigma$.
\end{itemize}
\end{definition}

A category with families supports $\Pi$- and $\Sigma$-types if it is
closed under the $\Pi$ and $\Sigma$ rules.

\section*{Presheaf model}

Let us now consider the categories of presheaves and see that they form a
category with families where the contexts are presheaves.  Let $\cat{C}$ be a
small category. We will refer to the objects of $\cat{C}$  as $I,J,K$ and
morphisms as $f,g,h$.

The context $\Gamma$ is then a presheaf on $\cat{C}$, $\Gamma \from \cat{C}^\op
\to \Set$, that is, we are given a set $\Gamma(I)$ for each $I$ in
$\ob(\cat{C})$, and functions $\Gamma(I) \to \Gamma(J)$, $\rho \mapsto \rho f$
(called \emph{restriction} and written as $f$ acting on the right) for each $f:
J \to I$ such that
\[
  \rho 1 = \rho \quad \text{and} \quad (\rho f) g = \rho (f g)
\]
for $g: K \to J$ and $f: J \to I$. We write $\Gamma_f$ for $\rho \mapsto \rho
f$.

The empty context $\diamond$ is a terminal presheaf and a context morphism
$\sigma \from \Delta \to \Gamma$ is a natural transformation that makes the
diagram commute.
\[
  \xymatrix{
    \Delta(I) \ar[r]^{\sigma_I} \ar[d]^{\Delta_f} & \Gamma(I) \ar[d]^{\Gamma_f} \\
    \Delta(J) \ar[r]^{\sigma_J} & \Gamma(J)
  }
\]

In the following, we will omit the indices on $\sigma$; Then if we write
$\Gamma_f$ and $\Delta_f$ as $f$ acting on the right (recall the session on
presheaves), the equation becomes:
\[
  (\sigma \rho) f = \sigma (\rho f)
\]
for $\rho$ in $\Delta(I)$.

Now we consider how to give a dependent type $\Gamma \vdash A~type$ in context
$\Gamma$. For each object $I \in \ob(\cat{C})$ and $\rho \in \Gamma(I)$ we require a
set $A \rho$, and for each $f \from J \to I$, we require a function $A \rho \to
A (\rho f)$, written as $a \mapsto af$ satisfying $a1 = a$ and $afg = a (f g)$
if $g \from K \to J$. The dependence on $I$ in $A\rho$ and on $I$ and $\rho$ in
$af$ is omitted for brevity. Substitution $\Delta \vdash A \sigma$ with $\sigma
\from \Delta \to \Gamma$ is simply given $(A \sigma) \rho = A (\sigma \rho)$
for $\rho \in \Gamma(I)$, together with the induced map
\[
  (A\sigma)\rho = A (\sigma \rho) \to A ((\sigma \rho)f) = A (\sigma (\rho f)) = (A \sigma) (\rho f)
\]
for $f \from J \to I$.

The definition of a dependent type can be rephrased more categorically: $A$ is
a presheaf on $\int_C \Gamma$, where $\int_C \Gamma$ is a category of elements
of the presheaf $\Gamma$, defined as follows:
\begin{itemize}
  \item objects are pairs $(I, \rho)$, where $I \in \ob(\cat{C})$ and $\rho \in
    \Gamma(I)$
  \item a morphism $(J, \rho') \to (I, \rho)$ is a morphism $f \from J \to I$
    in $\cat{C}$ such that $\rho' = \Gamma_f \rho$
\end{itemize}

A term $\Gamma \vdash a : A$ of a dependent type $\Gamma \vdash A~type$ is
given by a family of elements $a \rho \in A \rho$ for each $I \in \ob(\cat{C})$
and $\rho \in \Gamma(I)$, such that $a \rho f = a (\rho f)$ for each $f \from J
\to I$. The substitution $\Delta \vdash a \sigma : A \sigma$ with $\sigma \from
\Delta \to \Gamma$ is given by the family $(a \sigma) \rho = a (\sigma \rho)$.

For $\Gamma \vdash A~type$ the context extension $\vdash \Gamma.A~ctxt$ is
defined by
\begin{align*}
  &(\Gamma.A)(I) = \{(\rho, u)~|~\rho \in \Gamma(I) \land u \in A \rho\} &&
  \text{for $I \in \cat{C}$}\\
  &(\rho, u)f = (\rho f, u f) && \text{for $f \from J \to I$}
\end{align*}
and the projections are defined by $\mathbf{p} \from \Gamma.A \to \Gamma$,
$\mathbf{p}(\rho, u) = \rho$, and $\Gamma.A \vdash \mathbf{q} : Ap$ by
$\mathbf{q}(\rho, u) = u$.  Now assume $\sigma \from \Delta \to \Gamma$ and
$\Delta \vdash a : A\sigma$; we define $(\sigma, a) \from \Delta \to \Gamma.A$
by $(\sigma, a)\rho = (\sigma \rho, a \rho)$.

\subsection*{Dependent product}

As a motivation for dependent products recall how to define exponents in
presheaf categories. Let $\Gamma$ and $\Delta$ be presheaves and suppose we
already know how the exponent $\Delta^\Gamma$ is constructed. Then we get by
the Yoneda Lemma and the fact that $-^\Gamma$ should be right adjoint to $-
\times \Gamma$, that
\[
  (\Delta^\Gamma)(I) \cong Hom(\mathbf{y}I, \Delta^\Gamma) \cong Hom(\mathbf{y}I \times \Gamma, \Delta)
\]
where $\mathbf{y}$ is Yoneda embedding.

The latter can be taken up as a definition. So an element $w$ of
$(\Delta^\Gamma)(I)$ is a natural transformation $w \from \mathbf{y}I\times
\Gamma \to \Delta$, so it is given by functions
\[
  w_J \from (\mathbf{y}I)(J) \times \Gamma(J) \to \Delta(J),
\]
and the naturality condition becomes $(w_J(f, \rho))g = w_K(fg, \rho g)$ for $f
\from J \to I$, $\rho \in \Gamma(J)$, and $g \from K \to J$.

We will now show that CwF associated to a presheaf category supports
$\Pi$-types. Given $\Gamma \vdash A$ and $\Gamma.A \vdash B$ we have to define
$\Gamma \vdash \Pi x: A. B~type$, that is, we have to define the set $(\Pi x:
A.B) \rho$ for $I \in \cat{C}$ and $\rho \in \Gamma(I)$. The elements $w$ of
$(\Pi x:A.B)\rho$ are families $w = (w_f~|~J\in \cat{C}, f \from J \to I)$ of
(dependent) functions such that
\[
  w_f u \in B(\rho f, u) \qquad \text{for $J \in \cat{C}, f \from J \to I, and u \in A(\rho f)$},
\]
with the requirement that for $g \from K \to J$
\[
  (w_f u)g = w_{fg}(ug).
\]

For such a family $w \in (\Pi x: A. B)\rho$, the restriction $wf \in (\Pi
x:A.B)(\rho f)$ for $f: \from J \to I$ is defined by taking
\[
  (w f)_g u = w_{fg}u \in B(\rho f g, u)
\]
where $g \from K \to J$ and $u \in A(\rho f g)$. This definition satisfies $w 1
= w$ and $w f g = w(f g)$.

Now let $\Gamma.A \vdash b : B$; we have to define $\Gamma \vdash \lambda b :
\Pi x:A.B$, that is , give $(\lambda b)\rho \in (\Pi x:A.B)\rho$. For $f \from
J \to I$ and $u \in A(\rho f)$ we set
\[
  ((\lambda b)\rho)_f u = b (\rho f, u) \in B(\rho f, u)
\]
This satisfies for $g \from K \to J$
\[
  (((\lambda b)\rho)_f u)g = (b(\rho f, u))g = b(\rho(fg), ug) = ((\lambda b)\rho)_{fg}(ug)
\]
and defines a term since
\[
  (((\lambda b)\rho)f)_g u = ((\lambda b) \rho)_{fg} u = b(\rho fg, u) = ((\lambda b)(\rho f))_g u.
\]

To define the application let $\Gamma \vdash u : \Pi x:A.B$ and $\Gamma \vdash
v : A$, and we set $App(u, v)\rho = (u\rho)_1 (v\rho) \in B(\rho, v\rho)$, thus
$App(u, v)\rho \in B(1, v)\rho$ as $(\rho, v \rho) = (1, v)\rho$.
$\beta$-equality is readily checked
\[
  App(\lambda b, v)\rho = ((\lambda b)\rho)_1(v\rho)=b(\rho, v\rho)
\]
and similarly for $\eta$-equality and other equations.

\begin{example}
  Let us consider a term of type $A \to A$, and the elements in $Ty(\Gamma)$
  and $Ter(\Gamma, A \to A)$ for some $\vdash \Gamma~ctxt$, $\Gamma \vdash
  A~type$ in the presheaf model.

  For a CwF $(\cat{C}, F), I, J \in ob(\cat{C}), f \from J \to I$ we give the
  element of $\mathit{Ty}(\Gamma)$ corresponding to the type $A \to A$:
  \[
    (A \to A)\rho = \{w | w_f u \in A(\rho f), u \in A(\rho f)\} = \{\lambda I f u.u\}
  \]
  The required equalities hold trivially for any $g \from K \to J$, $((\lambda
  I f u.u)_f u) g = (\lambda I f u.u)_{fg} (ug)$.

  $\Gamma, x : A \vdash x : A$ corresponds to $q$ in the model. Thus $\Gamma
  \vdash \lambda x.x : A -> A$ corresponds to $\lambda q$ being defined as
  follows:
  \[
    ((\lambda q) \rho)_f u = q(\rho f, u) = u
  \]
\end{example}

\subsection*{Dependent sum}
The interpretation of $\Gamma \vdash \Sigma x:A.B~type$ for $\Gamma \vdash
A~type$ and $\Gamma.A \vdash B~type$ is defined by
\[
  (\Sigma x :A. B)\rho = \{(a, b)~|~a \in A\rho \land b \in B(\rho, a)\}
\]
for $\rho \in \Gamma(I), I \in \ob(\cat{C})$. The restrictions are defined
componentwise $(a, b)f = (af, bf)$ for $f \from J \to I$. The pairing operation
$\Gamma \vdash (u, v) : \Sigma x:A.B$ of $\Gamma \vdash u : A$ and $\Gamma.A
\vdash v : B(1, u)$ is defined by the componentwise pairing: $(u, v)\rho =
(u\rho, v\rho)$. Likewise for the projections $\mathbf{p}$ and $\mathbf{q}$,
$(\mathbf{p}w)\rho = a$ and $(\mathbf{q}w)\rho = b$ for $w\rho = (a, b)$. This
validates the necessary equations.

\section*{Credits}

Most of the material on the dependent types bases on \cite{Hofmann}, while
starting from categories with families we closely follow \cite{Huber}.

\bibliography{main}

\end{document}

% vim: ts=2 ss=2 sw=2 expandtab
