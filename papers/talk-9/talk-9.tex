\documentclass{article}
\def\pathToRoot{../../}\input{\pathToRoot headers/sharedHeader}

%\usepackage{todonotes}
\usepackage[utf8]{inputenc}
\usepackage{mathtools}
\usepackage{semantic}

\usetikzlibrary{cd, shapes, tikzmark}

\allowdisplaybreaks

\newtheorem{defn}{Definition}[section]
\newtheorem{exmpl}[defn]{Example}
\newtheorem{lemma}[defn]{Lemma}
\newtheorem{theorem}[defn]{Theorem}

\newcommand{\type}{\vdash}
\newcommand{\term}{1}
\newcommand{\unit}{\mathsf{unit}}
\newcommand{\app}{\mathsf{app}}
\newcommand{\cur}{\mathsf{cur}}
\newcommand{\fst}{\mathsf{fst}\,}
\newcommand{\snd}{\mathsf{snd}\,}
\newcommand{\id}{\mathsf{id}}
\newcommand{\eenv}{\diamond}


\title{Cartesian Closed Categories as a Model for the Simply Typed Lambda Calculus}
\author{Joachim Bard}

\begin{document}


\maketitle

\section{Motivation}
\label{sec:motiv}

Recall the Curry-Howard correspondence which relates logic and type theory.
According to this correspondence propositions are types and proofs are terms.
A prominent example is the relation between intuitionistic propositional logic and simply typed lambda calculus.
We can extend the correspondence to the so called Curry-Howard-Lawvere/Lambek correspondence by adding category theory as shown in the following table.

\begin{center}
    \begin{tabular}{l|l|l}
        Logic & Type theory & Category theory\\
        \hline
        Propositions & Types & Objects\\
        Proofs & Terms & Morphisms
    \end{tabular}
\end{center}

In the following sections we will draw a connection between the simply typed lambda calculus and cartesian closed categories.


\section{Simply Typed Lambda Calculus}

The simply types lambda calculus has types $A$ and terms $s$ as shown in the following definition.

\begin{defn}{(Simply Typed Lambda Calculus)}
    \label{def:stlc}
    \begin{align*}
        A, B & ::= G \mid \unit \mid A \times B \mid A \to B \\
        s, t & ::= x \mid c^A \mid () \mid (s, t) \mid \fst s \mid \snd s \mid \lambda x: A.\, s \mid s\, t
    \end{align*}
\end{defn}

Types consists of ground types $G$, the unit type $\unit$, product types $A \times B$ and function types $A \to B$.
Possible ground types are \textsf{int} and \textsf{bool}.
Terms are variables $x$, constants $c^A$ annotated by its type $A$, the empty tuple $()$, pairs $(s, t)$, first ($\fst s$) and second ($\snd s$) projections,
function definitions $\lambda x: A.\, s$ and function applications $s\,t$.
Common constants are integers like $3^{\mathsf{int}}$ and booleans ($\mathsf{true}^{\mathsf{bool}}$ and $\mathsf{false}^{\mathsf{bool}}$).
In order to state that a term has some type we use environments.
Environments are lists of variables and their types.
Thus we have the empty environment $\eenv$ and we can extend an environment $\Gamma$ with a variable $x$ of type $A$ (written $\Gamma, x: A$).
Often the empty environment is omitted.
For example the environment $x: \mathsf{int}, y: \mathsf{bool}$ has a variable $x$ of type \textsf{int} and a variable $y$ of type \textsf{bool}.
Variables in environments are always distinct, that means $x: \mathsf{int}, x: \mathsf{int}$ is not an environment.
We use environments to `remember' the type of a variable in a function definition.
The following definition shows how environments $\Gamma$ are used to state that a term $s$ has a type $A$.
We write $\Gamma \type s: A$ for this statement.
If the environment is empty we write just $\type s: A$ instead of $\eenv \type s: A$.
Also, we say that a term $s$ is well-typed, if an  environment $\Gamma$ and a type $A$ exists, such that the judgement $\Gamma \type s: A$ holds.

\begin{defn}{(Typing Rules)}
    \label{def:type}
    \begin{align*}
        &\qquad\qquad \inference{}{\Gamma, x: A \type x: A} \qquad \inference{\Gamma \type x : A}{\Gamma , y: B \type x: A}\\
        \\
        &\inference{}{\Gamma \type c^A: A} \qquad \inference{}{\Gamma \type () : \mathsf{unit}}\qquad
        \inference{\Gamma \type s: A \quad \Gamma \type t:B}{\Gamma \type (s, t) : A \times B}\\
        \\
        & \qquad\qquad \inference{\Gamma \type s : A \times B}{\Gamma \type \mathsf{fst}\,s: A} \qquad
        \inference{\Gamma \type s : A \times B}{\Gamma \type \mathsf{snd}\,s: B}\\
        \\
        & \quad \inference{\Gamma, x:A \type s: B}{\Gamma \type \lambda x: A.\, s: A \to B} \qquad
        \inference{\Gamma \type s: A \to B \quad \Gamma \type t: A}{\Gamma \type s\,t: B}
    \end{align*}
\end{defn}

We observe that the type of a variable is determined by looking into the environment.
The empty tuple $()$ is the only term of type $\unit$.
Furthermore, projections are only applicable to pairs.
As already mentioned above, the defined variable $x$ in a function definition $\lambda x: A.\, s$ is added to the environment
in order to specify the type of the body $s$.
But this directs to a problem if we consider the term $s := \lambda x: A.\,\lambda x: B.\,x$.
Applying the rules starting from the empty environment leads to $x: A, x: B \type x: B$.
The problem is that we have the variable $x$ twice in our `environment' $x: A, x: B$.
This is the reason why variables have to be distinct in environments.
To resolve this problem we consistently rename variables.
We rename the second $x$ to $y$ and obtain $s = \lambda x: A.\,\lambda y: B.\, y$.
Now $s$ has the type $A \to B \to B$, because we can derive $\type \lambda x: A.\,\lambda y: B.\, y: A \to B \to B$.
The consistent renaming of variables does not alter the term significantly.
Therefore, we always allow this renaming in the following sections
and call two terms $\alpha$-equivalent if one can be transformed into the other.


\section{Reminder: Cartesian Closed Category}

Before connecting cartesian closed categories and the simply typed lambda calculus in the next section,
we recall the definition of cartesian closed categories.

\begin{defn}
    \label{def:ccc}
    A cartesian closed category $\cat{C}$ consists of:
    \begin{itemize}
        \item A terminal object $\term \in \cat{C}$: $\forall X \in \cat{C}.\, \exists! \langle\rangle: X \to \term$.
        \item Binary products $X \times Y \in \cat{C}$ with projections $\pi_1 \from X \times Y \to X$ and $\pi_2 \from X \times Y \to Y$
            for all $X, Y \in \cat{C}$:
            For every $Z \in \cat{C}$ and morphisms $f \from Z \to X$ and $g \from Z \to Y$,
            there is exactly one morphism $\langle f, g \rangle \from Z \to X \times Y$, such that the following diagram commutes:
            \[
                \begin{tikzcd}
                    & Z \arrow[bend right]{ddl}{f} \arrow{d}{\langle f, g \rangle} \arrow[bend left]{ddr}{g} & \\
                    & X \times Y \arrow{dl}{\pi_1} \arrow{dr}{\pi_2} & \\
                    X & & Y
                \end{tikzcd}
            \]
            For two morphisms $p \from X \to X'$ and $q \from Y \to Y'$ we define
            \[ p \times q := \langle p \circ \pi_1, q \circ \pi_2 \rangle \from X \times Y \to X' \times Y'.\]
        \item Exponentials $Y^X \in \cat{C}$ together with a morphism $\app \from Y^X \times X \to Y$ for all $X, Y \in \cat{C}$:
            For every $Z \in \cat{C}$ and morphism $f \from Z \times X \to Y$,
            there is a unique morphism $\cur(f) \from Z \to Y^X$, such that following diagram commutes:
            \[
                \begin{tikzcd}
                    Z \times X \arrow{d}[left]{\cur(f) \times \id_X} \arrow{r}{f} & Y \\
                    Y^X \times X \arrow{ur}[right=7pt, below]{\app}
                \end{tikzcd}
            \]
    \end{itemize}
\end{defn}


\section{Semantics in a Cartesian Closed Category}
\label{sec:sem}

Let $\cat{C}$ be a cartesian closed category.
We want to express well-typed terms as morphisms in $\cat{C}$.
To that end, we first map types to objects of $\cat{C}$.
Thus, let $M$ be a function which maps every ground type $G$ to $M(G) \in \cat{C}$.
We can extend $M$ to arbitrary types:

\begin{align*}
    M|[G|] & := M(G) \\
    M|[\unit|] & := \term && (\text{terminal object}) \\
    M|[A \times B |] & := M|[A|] \times M|[B|] && (\text{product}) \\
    M|[A \to B |] & := M|[B|]^{M|[A|]} && (\text{exponential})
\end{align*}

In addition we map environments to objects of $\cat{C}$:

\begin{align*}
    M|[\eenv|] & := \term && (\text{terminal object}) \\
    M|[\Gamma, x: A|] & := M|[\Gamma|] \times M|[A|] && (\text{product})
\end{align*}

Furthermore, if $M$ maps every constant $c^A$ to a morphism $M(c^A) \from \term \to M|[A|]$,
then we define $M|[\Gamma \type s: A|] \from M|[\Gamma|] \to M|[A|]$ by recursion on $\Gamma \type s: A$ as follows.

\begin{defn}
    \label{def:type-C}
    \begin{align*}
        M|[\Gamma, x: A \type x: A|] & := M|[\Gamma|] \times M|[A|] \toby{\pi_2} M|[A|] \\
        M|[\Gamma, y: B \type x: A|] & := M|[\Gamma|] \times M|[B|] \toby{\pi_1} M|[\Gamma|] \xrightarrow{M\llbracket\Gamma \type x: A\rrbracket} M|[A|] \\
        M|[\Gamma \type c^A: A|] & := M|[\Gamma|] \toby{\langle\rangle} \term \xrightarrow{M(c^A)} M|[A|] \\
        M|[\Gamma \type (): \unit|] & := M|[\Gamma|] \toby{\langle\rangle} \term \\
        M|[\Gamma \type (s, t): A \times B|] & := M|[\Gamma|]
        \xrightarrow{\langle M\llbracket\Gamma \type s: A\rrbracket, M\llbracket\Gamma \type s: A\rrbracket\rangle} M|[A|] \times M|[B|] \\
        M|[\Gamma \type \fst s: A|] & := M|[\Gamma|] \xrightarrow{M\llbracket\Gamma \type s: A \times B\rrbracket} M|[A|] \times M|[B|] \toby{\pi_1} M|[A|] \\
        M|[\Gamma \type \snd s: A|] & := M|[\Gamma|] \xrightarrow{M\llbracket\Gamma \type s: A \times B\rrbracket} M|[A|] \times M|[B|] \toby{\pi_2} M|[B|] \\
        M|[\Gamma \type \lambda x: A.\, s: A \to B|] & := \cur(M|[\Gamma|] \times M|[A|] \xrightarrow{M\llbracket\Gamma, x: A \type s : B\rrbracket} M|[B|]) \\
        M|[\Gamma \type s\,t: B|] & := M|[\Gamma|]
        \xrightarrow{\langle M\llbracket\Gamma \type s: A \to B\rrbracket, M\llbracket\Gamma \type t: A\rrbracket\rangle}
        M|[B|]^{M|[A|]} \times M|[A|] \xrightarrow{\app} M|[B|]
    \end{align*}
\end{defn}

The case for constants illustrates why we choose $M(c^A) \from \term \to M|[A|]$.
Since the type of a constant does not depend on the environment $\Gamma$, it is natural to take the empty environment.
We can easily construct morphisms for arbitrary environments, because the empty environment is mapped to the terminal object.

\begin{exmpl}
    \label{exp:sem}
    \begin{align*}
            & M |[ \ \type \lambda f: A \to B.\, \lambda x: A.\, f\,x: (A \to B) \to A \to B |] \\
        = \,& \mathsf{cur}(M |[ f: A \to B \type \lambda x: A. \, f\,x : A \to B |]) \\
        = \,& \mathsf{cur}(\mathsf{cur}( M |[ f: A \to B, x: A \type f \, x : B |])) \\
        = \,& \mathsf{cur}(\mathsf{cur}(\mathsf{app} \circ (M |[\Gamma \type f: A \to B |], M |[\Gamma \type x : A|]))) \\
            &  \text{ where } \Gamma = f: A \to B, x: A \\
        = \,& \mathsf{cur}(\mathsf{cur}(\mathsf{app} \circ (M |[ f: A \to B \type f : A \to B |] \circ \pi_1, \pi_2))) \\
        = \,& \mathsf{cur}(\mathsf{cur}(\mathsf{app} \circ (M |[ f: A \to B \type f : A \to B |] \times \mathsf{id}))) && (\text{definition}~\ref{def:ccc}) \\
        = \,& \mathsf{cur}(M |[ f: A \to B \type f : A \to B |]) && (\text{definition}~\ref{def:ccc})\\
        = \,& M |[ \ \type \lambda f: A \to B.\, f: (A \to B) \to A \to B |]
    \end{align*}
\end{exmpl}


\section[beta-eta-equivalence]{$\beta\eta$-equivalence}

In the previous section we saw how to express well-typed terms as morphisms in a cartesian closed category.
Example~\ref{exp:sem} showed that there are terms which are mapped to the same morphism.
It turns out that this holds for all $\beta$- and $\eta$-equivalent terms.
In order to define this equivalence we need substitutions.
A substitution corresponds to a function application $(\lambda x: A.\, s)\, t$, where the function is called with an argument $t$.
It replaces in the term $s$ all occurrences of the variable $x$ with the term $t$.
We write $s[t/x]$ and define it as follows.

\begin{defn}{(Substitution)}
    \begin{align*}
        x[t/x] & := t\\
        y[t/x] & := y \qquad && y \neq x\\
        c^A[t/x] & := c^A\\
        ()[t/x] & := ()\\
        (s_1, s_2)[t/x] & := (s_1[t/x], s_2[t/x])\\
        (\mathsf{fst}\, s)[t/x] & := \mathsf{fst}(s[t/x])\\
        (\mathsf{snd}\, s)[t/x] & := \mathsf{snd}(s[t/x])\\
        (\lambda y:A.\, s)[t/x] & := \lambda y:A.\,(s[t/x]) \qquad && y \neq x \text{ and } y \text{ not in } t \\
        (s_1\, s_2)[t/x] & := (s_1[t/x])\,(s_2[t/x])
    \end{align*}
\end{defn}

The interesting case ($\lambda y: A.\, s)[t/x]$) is only applicable if $y \neq x$ and $y$ does not occur in $t$.
But because of $\alpha$-equivalence, we can always rename $y$ to a \emph{fresh} variable $z$ in order to ensure this condition.
\par
The next lemma illustrates how substitutions are interpreted in the category $\cat{C}$.

\begin{lemma}
    \label{lem:subst}
    If $\Gamma \type t: A$ and $\Gamma, x: A \type s: B$, then the diagram
    \[
        \begin{tikzcd}
            M \llbracket \Gamma \rrbracket \arrow{rr}{\langle \mathsf{id}, M\llbracket\Gamma \type t: A\rrbracket \rangle}
                                           \arrow{drr}[left=1cm, below]{M\llbracket\Gamma \type s[t/x] : B\rrbracket}
            & & M \llbracket\Gamma\rrbracket \times M \llbracket A \rrbracket \arrow{d}{M \llbracket\Gamma, x:A \type s: B\rrbracket} \\
            & & M \llbracket B \rrbracket
        \end{tikzcd}
    \]
    commutes.
\end{lemma}
\begin{proof}
    We omit the proof here.
    It can be found in~\cite{CT-STLC}.
\end{proof}

Using substitutions we can now define $\beta\eta$-equivalence.
If we read the $\beta$-conversions from left to right they correspond to computation steps.
Also the $\eta$-conversions relate terms, which have the same computational behavior.
The equivalence and congruence rules ensure that the $\beta\eta$-equivalence is an congruence relation.

\begin{defn}{($\beta\eta$-equivalence)}
    \begin{description}
        \item[$\beta$-conversions:]
            \begin{align*}
                & \qquad\qquad\qquad \inference{\Gamma, x:A \type s: B \quad \Gamma \type t:A}{\Gamma \type (\lambda x:A.\, s)\, t =_{\beta\eta} s[t/x]: B}\\
                \\
                & \quad \inference{\Gamma \type s: A \quad \Gamma \type t:B}{\Gamma \type \mathsf{fst}(s, t) =_{\beta\eta} s: A}\qquad
                \inference{\Gamma \type s: A \quad \Gamma \type t:B}{\Gamma \type \mathsf{snd}(s, t) =_{\beta\eta} t: B}
            \end{align*}
        \item[$\eta$-conversions:]
            \begin{align*}
                & \qquad\qquad\qquad \inference{\Gamma \type s: A \to B \quad x \text{ not in } s}{\Gamma \type s =_{\beta\eta} \lambda x: A.\,(s\,x): A \to B}\\
                \\
                & \inference{\Gamma \type s: A \times B}{\Gamma \type s =_{\beta\eta} (\mathsf{fst}\,s, \mathsf{snd}\,s): A \times B}\qquad
                \inference{\Gamma \type s: \mathsf{unit}}{\Gamma \type s =_{\beta\eta} () : \mathsf{unit}}
            \end{align*}
        \item[equivalence rules:]
            \begin{align*}
                & \inference{}{\Gamma \type s =_{\beta\eta} s: A} \qquad
                \inference{\Gamma \type s =_{\beta\eta} t: A}{\Gamma \type t =_{\beta\eta} s: A}
                \qquad \inference{\Gamma \type s =_{\beta\eta} t: A \quad \Gamma \type t =_{\beta\eta} u: A}{\Gamma \type s =_{\beta\eta} u: A}
            \end{align*}
        \item[congruence rules:]
            \begin{align*}
                & \inference{\Gamma \type s =_{\beta\eta} s': A \quad \Gamma \type t =_{\beta\eta} t': B}{\Gamma \type (s,t) =_{\beta\eta} (s',t'): A \times B}
                \qquad\inference{\Gamma \type s =_{\beta\eta} t : A \times B}{\Gamma \type \fst s =_{\beta\eta} \fst t: A} \\
                \\
                & \inference{\Gamma \type s =_{\beta\eta} t : A \times B}{\Gamma \type \snd s =_{\beta\eta} \snd t: B} \qquad
                \inference{\Gamma, x: A \type s =_{\beta\eta} t: B}{\Gamma \type \lambda x:A.\, s =_{\beta\eta} \lambda x:A.\, t: A \to B} \\
                \\
                & \qquad\qquad \inference{\Gamma\type s =_{\beta\eta} s': A \to B \quad \Gamma\type t =_{\beta\eta} t': A}{\Gamma\type s\,t =_{\beta\eta} s'\,t': B}
            \end{align*}
    \end{description}
\end{defn}

The following theorem shows that $\beta\eta$-equivalent terms are indeed mapped to the same morphism in a cartesian closed category.

\begin{theorem}{(Soundness)}
    \label{thm:sound}
    Let $\cat{C}$ be a cartesian closed category and
    let $M$ be a function which maps ground types to objects of $\cat{C}$ and constants to morphisms in $\cat{C}$ as required by section~\ref{sec:sem}.
    Every two terms $s$ and $t$ which are $\beta\eta$-equivalent of type $A$ under environment $\Gamma$ lead to equal morphisms in $\cat{C}$, i.\,e.
    if $\Gamma \type s =_{\beta\eta} t: A$, then $M|[\Gamma \type s: A|] = M|[\Gamma \type t: A|]$.
\end{theorem}
\begin{proof}
    The proof goes by induction on $\Gamma \type s =_{\beta\eta} t: A$.
    We will show the case for
    \[\Gamma \type (\lambda x:A.\, s)\, t =_{\beta\eta} s[t/x]: B.\]
    Let $f := M|[\Gamma, x: A \type s: B|]$ and $g := M|[\Gamma \type t: A|]$.
    By definition~\ref{def:type-C} we get \[\cur(f) = M|[\Gamma \type \lambda x: A.\, s: A \to B|].\]
    Using this we obtain
    \begin{align*}
        M|[\Gamma \type (\lambda x: A.\, s)\, t: B|] & = \app \circ \underbrace{\langle \cur(f), g \rangle}_{= (\cur(f) \times \mathsf{id}) \circ \langle \mathsf{id}, g \rangle} && (\text{definition}~\ref{def:type-C}) \\
                                                     & = \underbrace{\app \circ (\cur(f) \times \mathsf{id})}_{= f} \circ \langle \mathsf{id}, g \rangle && (\text{definition}~\ref{def:ccc})\\
                                                     & = f \circ \langle \mathsf{id}, g \rangle && (\text{definition}~\ref{def:ccc})\\
                                                     & = M|[\Gamma \type s[t/x]|] && (\text{lemma}~\ref{lem:subst})
    \end{align*}
\end{proof}


\section{Internal Language of a Cartesian Closed Category}

In the previous section we established a connection between $\beta\eta$-equivalent terms and equal morphisms.
We can exploit this connection to use the simply type lambda calculus to describe cartesian closed categories.
To do so, let $\cat{C}$ be a cartesian closed category.
We want to define a simply type lambda calculus which resembles $\cat{C}$, which we call internal language of $\cat{C}$.
Therefore, we take every object $G$ of $\cat{C}$ as ground type.
Furthermore, for every morphism $f \from \term \to A$, where $\term$ is the terminal object, we have a constant $f^A$ of type $A$.
The mapping from section~\ref{sec:sem} is defined as $M(G) := G$ for every ground type $G$ and $M(f^A) := f$ for every constant $f^A$.
The following example illustrates how the internal language can be used to show properties of $\cat{C}$.

\begin{exmpl}
    We show that $C^{A \times B} \iso (C^B)^A$ for all objects $A, B$ and $C$ from $\cat{C}$.
    Note that $A, B$ and $C$ are types in the internal language of $\cat{C}$.
    First, we  define lambda terms $cas$ and $car$:
    \begin{align*}
        & cas := \lambda f: A \times B \to C.\,\lambda x: A.\,\lambda y: B.\,f(x, y) \\
        & car := \lambda g: A \to B \to C.\,\lambda p: A \times B.\,g\,(\mathsf{fst}\,p)\,(\mathsf{snd}\,p),
    \end{align*}
    such that they satisfy the following typing judgements and $\beta\eta$-equivalences:
    \begin{align*}
        & \vdash cas: (A \times B \to C) \to A \to B \to C \\
        & \vdash car: (A \to B \to C) \to A \times B \to C\\
        & f: A \times B \to C \vdash car\, (cas\, f) =_{\beta\eta} f : A \times B \to C \\
        & g: A \to B \to C \vdash cas\, (car\, g) =_{\beta\eta} g : A \to B \to C.
    \end{align*}
    Furthermore, we define
    \begin{align*}
        h & := M|[f: A \times B \to C \vdash cas\,f: A \to B \to C|] \\
        h' & := M|[g: A \to B \to C \vdash car\,g: A \times B \to C|].
    \end{align*}
    Note that $\langle\rangle \circ h = \pi_1$ and thus
    \begin{align*}
        (\mathsf{cur}(h') \times \id_{\term}) \circ \langle \langle\rangle, \id_{(C^B)^A} \rangle \circ h = \langle \mathsf{cur}(h') \circ \pi_1, h \rangle. \tag{1}
    \end{align*}
    We obtain
    \begin{align*}
        & h' \circ \langle\langle\rangle, \id_{(C^B)^A}\rangle \circ h \\
        = \,& \mathsf{app} \circ (\mathsf{cur}(h') \times \id_\term) \circ \langle\langle\rangle, \id_{(C^B)^A}\rangle \circ h && (\text{definition}~\ref{def:ccc})\\
        = \,& \mathsf{app} \circ \langle \mathsf{cur}(h') \circ \pi_1, h \rangle && (\text{equation}~(1))\\
        = \,& \mathsf{app} \circ \langle M|[\ \vdash car: (A \to B \to C) \to A \times B \to C|] \circ \pi_1, h \rangle && (\text{definition}~\ref{def:type-C})\\
        = \,& \mathsf{app} \circ \langle M|[f: A \times B \to C \vdash car: (A \to B \to C) \to A \times B \to C|] , h \rangle && (\text{definition}~\ref{def:type-C})\\
        = \,& M|[f: A \times B \to C \vdash car\,(cas \,f): A \times B \to C|] && (\text{definition}~\ref{def:type-C}) \\
        = \,& M|[f: A \times B \to C \vdash f: A \times B \to C|] && (\text{theorem}~\ref{thm:sound})\\
        = \,& \pi_2 && (\text{definition}~\ref{def:type-C})
    \end{align*}
    and therefore
    \begin{align*}
        & h' \circ \langle\langle\rangle, \id_{(C^B)^A}\rangle \circ h \circ \langle \langle\rangle, \id_{C^{A \times B}} \rangle \\
        =\, & \pi_2 \circ \langle \langle\rangle, \id_{C^{A \times B}} \rangle \\
        =\, & \id_{C^{A \times B}} && (\text{definition}~\ref{def:ccc}).
    \end{align*}
    Analogously we obtain
    \[ h \circ \langle \langle\rangle, \id_{C^{A \times B}} \rangle \circ h' = \pi_2 \tag{2} \]
    and thus
    \begin{align*}
        & h \circ \langle \langle\rangle, \id_{C^{A \times B}} \rangle \circ h' \circ \langle\langle\rangle, \id_{(C^B)^A}\rangle \\
        =\, & \pi_2 \circ \langle\langle\rangle, \id_{(C^B)^A}\rangle && (\text{equation}~(2))\\
        =\, & \id_{(C^B)^A} && (\text{definition}~\ref{def:ccc}).
    \end{align*}
    This shows $C^{A \times B} \iso (C^B)^A$.
\end{exmpl}


%\iffalse
\section{Free Cartesian Closed Categories}
\label{sec:free}

The soundness theorem~\ref{thm:sound} has a converse called completeness.

\begin{theorem}{(Completeness)}
    \label{thm:compl}
    If $M|[\Gamma\type s: A|] = M|[\Gamma\type t: A|]$ for every mapping $M$ of ground types and constants and every cartesian closed category $\cat{C}$,
    then $\Gamma \type s =_{\beta\eta} t: A$.
\end{theorem}
\begin{proof}
    We give a proof sketch.
    The idea is to construct a freely generated cartesian closed category $\cat{F}$ and instantiate the assumption with $\cat{F}$.
    Objects of $\cat{F}$ are types of the simply typed lambda calculus.
    The set of morphisms from $A$ to  $B$ is the quotient of the set $\{s \mid \ \type s: A \to B\}$, where two terms $s$ and $t$ are equivalent iff
    $\ \type s =_{\beta\eta} t: A \to B$ holds.
    Showing that $\cat{F}$ is a cartesian closed category concludes the proof.
\end{proof}
%\fi


\nocite{*}
\bibliographystyle{plain}
\bibliography{bib}

\end{document}
